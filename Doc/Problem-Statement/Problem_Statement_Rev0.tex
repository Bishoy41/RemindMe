\documentclass[a4paper]{article}
\usepackage[utf8x]{inputenc}
\usepackage{amsmath}
\usepackage[english]{babel}
\usepackage[a4paper, top=3cm, bottom=2cm, left=3cm, right=3cm, marginparwidth=1.75cm]{geometry}

% Title Page
\title{SE 3XA3: Reminders Problem Statement}
\date{\today}
\author{
    Group 24 - L02,
    \\ Jenkins, Michael
    \\ Panunto, Michael
    \\ Sefin, Bishoy
}

\begin{document}
\maketitle

% Turning off section numbering
\setcounter{secnumdepth}{0}

\newpage

% Summary of the problem, not the solution we are constructing
\section{What is the problem being addressed?}

In a world reliant on social media, a significant portion of one's life is spent online in constant interaction with others. With this lifestyle an issue arises; One becomes so busy and caught up in living between their personal and online lives that many plans, ideas, commitments are forgotten. Additionally, how does one hold onto the special memories and interactions that happen on a daily basis but are constantly being buried beneath a swarm of new updates?

% Why does the problem matter?
\section{Significance of the problem?}

Both issues presented above have a significant impact in daily life for a multitude of people. With such busy lives, many tasks are forgotten each day affecting both professional and personal aspects of one's life. Whether it's forgetting to send an important report to a supervisor before leaving the office or not calling a parent on a special event, such a simple oversight can develop issues within one's career and relationships. 
~\\~\\
The issue that inspired the original project also demonstrates the significance of such a problem. When a family member had unfortunately passed away, there was a wife and two young boys left behind. While we can't always prevent such unexpected events to take place, we can leave more of ourselves behind in the case that it does happen. By replying to the reminder email every day, week, month etc... a small piece of one's life from that moment is left behind, documented permanently for others close to them to easily access and hold onto in the case of such an unexpected tragedy. 

% Stakeholders, software environment etc...
\section{Problem context?}

While the application is targeted at a wide range of users, the main focus consists of students and those within the work force as they are benefitted the most.
~\\~\\
With the original project being a google application, we plan to turn it into a mobile application running on Android through the use of Java and XML coding. 
\end{document}


