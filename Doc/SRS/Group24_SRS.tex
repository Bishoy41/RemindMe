\documentclass[12pt, titlepage, a4paper]{article}
\usepackage[utf8x]{inputenc}
\usepackage[english]{babel}
\usepackage[a4paper, top=3cm, bottom=2cm, left=3cm, right=3cm, marginparwidth=1.75cm]{geometry}
\usepackage[table]{xcolor}
\usepackage{float}
\usepackage[colorlinks=true, allcolors=blue]{hyperref}

\title{SE 3XA3: Software Requirements Specification}
\author{
    Reminders - Group 24 - L02\\
    Michael Panunto - panuntom, \\
    Michael Jenkins - jenkinma, \\
    Bishoy Sefin - sefinb
}
\date{\today}

\begin{document}
% Options to be used for table
\setlength{\arrayrulewidth}{1.5pt}
\definecolor{light-gray}{gray}{0.90}

\maketitle

\pagenumbering{roman}

% Revision history table
\begin{table}[H]
    \begin{center}
        \rowcolors{1}{white}{light-gray}
        \caption{Revision History} \label{tab:RevisionHistory}
        \begin{tabular}{l p{0.4\linewidth} p{0.4\linewidth}}
            \hline
            \textbf{Date} & \textbf{Developer(s)} & \textbf{Change} \\
            \hline
            2017-10-06 & Michael Panunto, \newline Michael Jenkins, \newline Bishoy Sefin & Created SRS - rev. 0 \\
            \hline 
        \end{tabular}
    \end{center}
\end{table}

\tableofcontents

\newpage
\pagenumbering{arabic}
% General info about the project
\section{Project Drivers}

\subsection{Purpose of the Project}
This project is designed to allow users to remember the smaller important moments that occur in day to day life, while additionally notifying these users on important events coming up.

% We need to work on stakeholders
\subsection{Stakeholders}
The primary stakeholder for Reminders are students and members of the work force. Their interest in the project stems from a busy lifestyle where balancing personal lives and work/school is becoming increasingly more difficult.

% Functional Reqs - Should be numbered
\section{Functional Requirements}
\begin{enumerate}
    \item The application shall run on the Android operating system.
    \item The application shall use e-mail permissions to send e-mail reminders to the client.
    \item A GUI shall be displayed upon application launch with options  to create/edit a reminder or view response history.
    \item The user shall be able to set a one time, or scheduled reminder.
    \item The application will store responses to the reminder e-mails.
    \item The application shall send e-mails through the user's specified application.
    \item The application will detect if no suitable e-mail application services are available, and display a related message.
    \item The application shall have a password feature available to the user.
\end{enumerate}

% Non-functional Reqs
\section{Non-functional Requirements}
\begin{itemize}
    \item Accessibility: The application shall have a minimal interface, easily accessible to users of little Android experience.
    \item Performance: 
    \begin{itemize}
        \item The application will create reminders quickly so as to not interfere with the user's other tasks. 
        \item Memory usage from the app must not affect performance of other applications
    \end{itemize}
    \item Security: The application must offer safe-keeping of reminder responses.
    \item Operational Constraints: The application must work on Android (It will not work on other operating systems). The Android version must be at least KitKat 4.4.
    \item Physical Constraints: The application must not use more memory than available on low-end Android devices.
\end{itemize}

\section{Project Issues}

\subsection{Off-the-Shelf Solutions}
Countless apps to notify users currently exist on the marketplace, and even come standard on most devices. However, Reminders separates from the rest by sending notifications through e-mail, and storing responses to the e-mails so that moments each day, week, month etc... Can be recorded and looked back on at any point in the future.

\subsection{Problems in Current Environment}
The main problem existing within current implementation of the project is the accessibility. Users currently have to access it from a browser, where the application isn't optimized for phones making it difficult to set reminders on the go. Converting the application to android offers a solution to this problem, as it makes it easier to use when not near a computer.

\end{document}